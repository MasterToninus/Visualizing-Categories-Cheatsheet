%% A tentative, eye-catchy, one-page definition of category (I would like to replicate the page frome my handwritten notes.
\documentclass[preview]{standalone}

\usepackage{amsmath}
\usepackage{verbatim}
\usepackage[italian,english]{babel}
\usepackage[utf8]{inputenc}

\usepackage[basic,cat]{./Math-Symbols-List/toninus-math-symbols}
\usepackage{./Latex-Theorem/theoremtemplate}
\usepackage{./visualcat}
\usepackage{pdflscape}




\begin{document}
%
The best way to understand what is a limiti is through examples on a "model" categoty. The prototypical complete and co-complete category is $\cat[Set]$


\begin{minipage}[t]{0.5\textwidth}
\begin{example}[]
	Set has terminal object:
	
\end{example}
\end{minipage}
\vrule{}
\begin{minipage}[t]{0.5\textwidth}
\begin{example}[Initial Object]
	Is a colimit on the on the empty diagram .\\
	$\textcolor{red}{I\in \Obj(\cat)} \St 
	 \textcolor{blue}{\forall X \in \Obj	(\cat)}
	 \textcolor{orange}{\exists! I\rightarrow X}$\\


	\begin{tikzcd}
		\textcolor{red}{I} \ar[d,orange,"\exists!"'] \\
		\textcolor{blue}{X} 
	\end{tikzcd}
\end{example}
\end{minipage}



\end{document}
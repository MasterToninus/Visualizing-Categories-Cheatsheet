%% A tentative, eye-catchy, one-page definition of category (I would like to replicate the page frome my handwritten notes.
\documentclass[preview]{standalone}

\usepackage{amsmath}
\usepackage{verbatim}
\usepackage[italian,english]{babel}
\usepackage[utf8]{inputenc}

\usepackage[basic,cat]{./Math-Symbols-List/toninus-math-symbols}
\usepackage{./Latex-Theorem/theoremtemplate}
\usepackage{./visualcat}
\usepackage{pdflscape}

\def\shapeproduct{
	\tikz[baseline=.1ex]{
		\fill (0,1ex) circle [radius=2pt] coordinate (A);
		\fill (2.5ex,1ex) circle [radius=2pt] coordinate (B);
	}
}
\def\shapeequalizer{
	\tikz[baseline=.1ex]{
		\fill (0,1ex) circle [radius=2pt] coordinate (A);
		\fill (3ex,1ex)circle [radius=2pt] coordinate (B);
		\fill (0.4ex,1.5ex) coordinate (Aup);
		\fill (0.4ex,0.5ex) coordinate (Adown);
		\fill (2.6ex,1.5ex) coordinate (Bup);
		\fill (2.6ex,0.5ex) coordinate (Bdown);
		\draw[black,->] (Aup) -- (Bup);
		\draw[black,->] (Adown) -- (Bdown);
	}
}
\def\shapepullback{
	\tikz[baseline=.1ex]{
		\fill (0ex,-0.5ex) circle [radius=1.5pt] coordinate (A);
		\fill (3ex,-0.5ex)circle [radius=1.5pt] coordinate (B);
		\fill (3ex,2.5ex) circle [radius=1.5pt] coordinate (C);
		\draw[black,->,shorten >=0.4ex,shorten <=0.4ex] (A) -- (B);
		\draw[black,->,shorten >=0.4ex,shorten <=0.4ex] (C) -- (B);
	}
}
\def\shapepushout{
	\tikz[baseline=.1ex]{
		\fill (0,0) circle [radius=1.5pt] coordinate (A);
		\fill (0ex,3ex)circle [radius=1.5pt] coordinate (B);
		\fill (3ex,3ex) circle [radius=1.5pt] coordinate (C);
		\draw[black,->,shorten >=0.4ex,shorten <=0.4ex] (B) -- (A);
		\draw[black,->,shorten >=0.4ex,shorten <=0.4ex] (B) -- (C);
	}
}



\begin{document}
%
The best way to grasp the meaning (...) to understand what is a limit it is through examples. The model category is $\cat[Set]$ which turn out to be complete and co-complete.



\begin{minipage}[t]{0.5\textwidth}
\begin{definition}[Terminal Object]
	Is a limit on the empty diagram .\\
	$\textcolor{red}{T\in \Obj(\cat)} \St 
	 \textcolor{blue}{\forall X \in \Obj	(\cat)}
	 \textcolor{orange}{\exists! X\rightarrow T}$\\
	 
	\begin{tikzcd}
		\textcolor{blue}{X} \ar[d,orange,"\exists!"'] \\
		\textcolor{red}{T} 
	\end{tikzcd}
\end{definition}
\end{minipage}
\vrule{}
\begin{minipage}[t]{0.5\textwidth}
\begin{definition}[Initial Object]
	Is a colimit on the on the empty diagram .\\
	$\textcolor{red}{I\in \Obj(\cat)} \St 
	 \textcolor{blue}{\forall X \in \Obj	(\cat)}
	 \textcolor{orange}{\exists! I\rightarrow X}$\\


	\begin{tikzcd}
		\textcolor{red}{I} \ar[d,orange,"\exists!"'] \\
		\textcolor{blue}{X} 
	\end{tikzcd}
\end{definition}
\end{minipage}

\begin{minipage}[t]{0.5\textwidth}
\begin{definition}[Product]
	Is a limit on the shape \shapeproduct .\\
	\begin{tikzcd}
		&\textcolor{blue}{C} \ar[ddl, bend right, blue] \ar[ddr, bend left, blue] \ar[d,orange,"\exists!"'] 
		\ar[ddl,phantom,orange, "\commute"] \ar[ddr,phantom,orange, "\commute"]& \\
		&\textcolor{red}{A \times B} \ar[dl,red] \ar[dr,red]& \\
		A & & B \\
	\end{tikzcd}
\end{definition}
\end{minipage}
\vrule{}
\begin{minipage}[t]{0.5\textwidth}
\begin{definition}[Co-Product]
	Is a colimit on the shape \shapeproduct .\\
	\begin{tikzcd}
		 A \ar[dr,red] \ar[ddr, bend right, blue] & &  B\ar[dl,red] \ar[ddl, bend left, blue]\\
		&\textcolor{red}{A \times B}  \ar[d,orange,"\exists!"'] & \\
		&\textcolor{blue}{C} \ar[uul,phantom,orange, "\commute"] \ar[uur,phantom,orange, "\commute"]& 
	\end{tikzcd}
\end{definition}
\end{minipage}

\begin{minipage}[t]{0.5\textwidth}
\begin{definition}[Equalizer]
	Is a limit on the shape \shapeequalizer .\\
	\begin{tikzcd}
		& & \phantom. & B \ar[dd,equal] \\
		\textcolor{blue} C \ar[rrrd,blue, bend right,looseness=1]\ar[rrru,blue,bend left]\ar[r,orange,"\exists!"]\ar[drr,phantom,orange, "\commute"]\ar[urr,phantom,orange, "\commute"]
		& \textcolor{red} E \ar[r,red,"e"] \ar[rru,red, bend left] \ar[rrd,red, bend right] \ar[rrd,phantom,red, "\commute"] \ar[rru,phantom,red, "\commute"] & 
		A \ar[ur,"f"] \ar[dr,"g"'] \ar[r,phantom, "\commute"]& \phantom. \\
		& \phantom. & \phantom. & B 
	\end{tikzcd}
\end{definition}
\end{minipage}
\vrule{}
\begin{minipage}[t]{0.5\textwidth}
\begin{definition}[Co-Equalizer]
	Is a colimit on the shape \shapeequalizer .\\
	\begin{tikzcd}
		A   \ar[dd,equal] \ar[rd,"f"] \ar[rrd,red, bend left] \ar[rrd,phantom,red, "\commute"]& \phantom. & &\\
		\phantom. \ar[r,phantom, "\commute"] & B \ar[r,red,"q"] & \textcolor{red} Q & \textcolor{blue}C 
		\ar[llld,blue, bend left,looseness=1,leftarrow]\ar[lllu,blue,bend right,leftarrow]\ar[l,orange,"\exists!"',leftarrow]\ar[ull,phantom,orange, "\commute"]\ar[dll,phantom,orange, "\commute"]		
		\\
		B \ar[ru,"g"'] \ar[rru,red, bend right] \ar[rru,phantom,red, "\commute"]& \phantom. & & \\
	\end{tikzcd}
\end{definition}
\end{minipage}

\begin{minipage}[t]{0.5\textwidth}
\begin{definition}[Pull-Back]
	Is a limit on the shape \shapepullback .\\
	\begin{tikzcd}
		\textcolor{blue} C \ar[rd,orange,"\exists!"]\ar[rdd,phantom,orange, "\commute"]\ar[rrd,phantom,orange, "\commute"]   \ar[rdd,blue,bend right] \ar[rrd,blue,bend left]& & \\
		& \textcolor{red} S \ar[rd,red,phantom,"\commute"] \ar[r,red] \ar[d,red]  \arrow[dr, phantom, "\ulcorner", very near start]
& A \ar[d] \\
		& B \ar[r] & C
	\end{tikzcd}
\end{definition}
\end{minipage}
\vrule{}
\begin{minipage}[t]{0.5\textwidth}
\begin{definition}[Push-Out]
	Is a colimit on the shape \shapepushout .\\
	\begin{tikzcd}
		C \ar[r] \ar[d] \ar[rd,red,phantom,"\commute"]& A \ar[d,red] \ar[rdd,blue,bend left]& \\
		B  \ar[r,red] \ar[rrd,blue,bend right] & \textcolor{red}O \arrow[ul, phantom, "\ulcorner", very near start] \ar[rd,orange,"\exists!"] & \\
		& & \textcolor{blue} U	\ar[llu,phantom,orange, "\commute"]\ar[luu,phantom,orange, "\commute"]	
	%	 U   \ar[rdd,green,bend right] \ar[rrd,green,bend left]& & \\
	%	& } S  \arrow[dr, phantom, "\ulcorner", very near start] & A \ar[d] \\
	%	& B \ar[r] & C
	\end{tikzcd}
\end{definition}
\end{minipage}

So, a limit is a concept pertaing to a specific configuration of objects and arrows (the image of the diagram $D$ in $\cat$).
\\
We reserve a special name for categories such that limit exists for all diagram of a certain shape regardless of the specificity of vertex and arrows (i.e. depending only on the shape of the diagrams).
%
\begin{notation}
	\begin{displaymath}
		D: \cat[I] \rightarrow	\cat
	\end{displaymath}
	A diagram is \emph{finite} if $\Obj(\cat[I])$ is a discrete set.
	\\
	Limits on a finite diagram are called \emph{finite limits}.
\end{notation}
%
\begin{definition}[(finitely) (Co-)Complete Category]
	$\cat$ \St all (finite) (Co-)Limit exists.
\end{definition}
%
\begin{proposition}
T.f.s.a.e.:
\begin{itemize}
	\item $\cat$ has all (finite) limits / (finite) colimits
	\item $\cat$ has all (finite) products and equalizers / (finite) co-products and co-equalizers
	\item $cat$ has all (binary) pullbacks and terminal object / (binary) pushout and initial object
\end{itemize}

\end{proposition}

\end{document}
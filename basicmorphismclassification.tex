%% A tentative, eye-catchy, one-page definition of category (I would like to replicate the page frome my handwritten notes.
\documentclass[preview]{standalone}

\usepackage{amsmath}
\usepackage{verbatim}
\usepackage[italian,english]{babel}
\usepackage[utf8]{inputenc}

\usepackage[basic,cat]{./Math-Symbols-List/toninus-math-symbols}
\usepackage{./Latex-Theorem/theoremtemplate}
\usepackage{./visualcat}

\usepackage{tikz}
\def\tria{
	\tikz[baseline=.1ex]{
		\fill (0,-0.5ex) coordinate (A);
		\fill (0,2ex) coordinate (B);
		\fill (2.25ex,0.75ex) coordinate (C);
		\draw[green] (0.75ex,0.75ex) node {$\commute$};
		\draw[blue] (A) -- (B);
		\draw[green] (A) -- (C) -- (B);
	}
}

\def\trib{
	\tikz[baseline=.1ex]{
		\fill (0,-0.5ex) coordinate (A);
		\fill (0,2ex) coordinate (B);
		\fill (2.25ex,0.75ex) coordinate (C);
		\draw[blue] (0.75ex,0.75ex) node {$\commute$};
		\draw[blue] (A) -- (B);
		\draw[blue] (A) -- (C) -- (B);
	}
}

\def\cotria{
	\tikz[baseline=.1ex]{
		\fill (0,-0.5ex) coordinate (A);
		\fill (0,2ex) coordinate (B);
		\fill (-2.25ex,0.75ex) coordinate (C);
		\draw[green] (-0.75ex,0.75ex) node {$\commute$};
		\draw[blue] (A) -- (B);
		\draw[green] (A) -- (C) -- (B);
	}
}

\def\cotrib{
	\tikz[baseline=.1ex]{
		\fill (0,-0.5ex) coordinate (A);
		\fill (0,2ex) coordinate (B);
		\fill (-2.25ex,0.75ex) coordinate (C);
		\draw[blue] (-0.75ex,0.75ex) node {$\commute$};
		\draw[blue] (A) -- (B);
		\draw[blue] (A) -- (C) -- (B);
	}
}

\begin{document} 
% first column
\begin{minipage}[t]{0.5\textwidth}
	\begin{definition}[Monomorphism (mono or monic arrow)]
		\emph{"left cancellative arrow"}
		$\textcolor{red}{\bulk} s.t. \textcolor{blue}{\forall \bulk} \quad \tria \Rightarrow \trib$\\
		%
		\begin{tikzcd}
			 \textcolor{blue}{B} \ar[dd,blue,equal] \ar[dr,blue,"g_1"] \arrow[rrd, bend left=30, "fg_1",green] & & \\
			\phantom. & \textcolor{red}{C}  \ar[r,red,"f",tail]& \textcolor{red}{D}  \\
			\textcolor{blue}{B}  \ar[ur,blue,"g_2"] \arrow[rru, bend right=30, "fg_2",green]& & 
		\end{tikzcd}
		\footnote{$(f g_1 = f g_2 ) \Rightarrow ( g_1 = g_2 )$}
	\end{definition}
	(Hom functor point of view:)
	\begin{displaymath}
		f \textrm{mono} \Leftrightarrow H^B(f) \textrm{injective} \quad \forall B \in \cat
	\end{displaymath}
	%
	\begin{tikzpicture}[baseline= (a).base]
	\node[scale=.75] (a) at (0,0){
		\begin{tikzcd}
			C \ar[d,"f"] & & \hom(B,C) \ar[d, "H^B(f) = f(-)"]\ar[r,phantom,"\in"] & g_1 \ar[d,mapsto]\ar[r,phantom,','] & g_2 \ar[d,mapsto] \\
			D				 & & \hom(B,D) \ar[r,phantom,"\in"] & f g_1 \ar[r,phantom,"="] & f g_2 	
		\end{tikzcd}
	};
	\end{tikzpicture}
	%
	\begin{example}
	(Mono "generalize" (there are nuance of monomorphism) the concept of injective map)\\
	Monomorphisms in $Set$ are injective functions. (It's straightforward from the definition via hom functors taking $B= {\ast}$)
	\begin{tikzpicture}[baseline= (a).base]
	\node[scale=.75] (a) at (0,0){
		\begin{tikzcd}
			\{\ast\} \ar[dd,equal] \ar[dr,"c_1"] & & \\
			& C \ar[r,tail,"f"] & D \\
			\{\ast\} \ar[ur,"c_2"]& &
		\end{tikzcd}
	};
	\end{tikzpicture}
	\\
	Mono means:
	\begin{displaymath}
		f(c1) = f(c2) \Rightarrow c1=c2
	\end{displaymath}
	that is injectivity
	\end{example}
 
\end{minipage}
\vrule{}
\begin{minipage}[t]{0.5\textwidth}
	\begin{definition}[Epimorphism (epi or epic arrow)]
		\emph{"right cancellative arrow"}
		$\textcolor{red}{\bulk} s.t. \textcolor{blue}{\forall \bulk} \quad \cotria \Rightarrow \cotrib$
		
		\begin{tikzcd}
			& & \textcolor{blue}{E} \ar[dd,blue,equal] \\
			\textcolor{red}{C} \ar[r,red,"f",twoheadrightarrow] \arrow[rru, bend left=30, "h_1 f",green] \arrow[rrd, bend right=30, "h_2 f",green]& \textcolor{red}{D} \ar[ur,blue,"h_1"] \ar[dr,blue,"h_2"] & \\
			& & \textcolor{blue}{E}
		\end{tikzcd}
		\footnote{$(h_1 f = h_2 f ) \Rightarrow ( h_1 = h_2 )$}
	\end{definition}

	(Hom functor point of view:)
	\begin{displaymath}
		f \textrm{epi} \Leftrightarrow H_E(f) \textrm{injective} \quad \forall E \in \cat
	\end{displaymath}	
			\begin{tikzpicture}[baseline= (a).base]
			\node[scale=.75] (a) at (0,0){
				\begin{tikzcd}
					C \ar[d,"f"] & & \hom(C,E) \ar[r,phantom,"\in"] & h_1 f \ar[r,phantom,"="] & h_2 f  \\
					D				 & & \hom(D,E) \ar[u, "H_E(f) = (-)f"]  \ar[r,phantom,"\in"] & h_1 \ar[u,mapsto] \ar[r,phantom,','] & h_2 \ar[u,mapsto]	
				\end{tikzcd}
			};
			\end{tikzpicture}
	%
	\begin{example}
	(Epi "generalizes" (there are nuance of monomorphism) the concept of surjective map)\\
	Epimorphisms in $Set$ are surjective functions. \\
	Consider
	\begin{tikzpicture}[baseline= (a).base]
	\node[scale=.75] (a) at (0,0){
		\begin{tikzcd}
			& & \{0,1\} \ar[dd,equal]\\
		X	\ar[r,"f",twoheadrightarrow]&Y \ar[ur,"\chi_{f(x)}"]\ar[dr,"1"] & \\
			&	 & \{0,1\}
		\end{tikzcd}
	};
	\end{tikzpicture}
	\\
	where $\chi_{f(x)}$ is the characteristic function on the range $f(X) \subset Y$ and $1$ is the constant function.\\
	Epi means:
	\begin{displaymath}
		\chi_{f(x)} = 1 \Rightarrow \chi_{f(X)} = 1 \Rightarrow Y = f(X)
	\end{displaymath}
	that is surjectivity
	\end{example}
	
\end{minipage}






\end{document}

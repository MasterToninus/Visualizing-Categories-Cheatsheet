%% A tentative, eye-catchy, one-page definition of category (I would like to replicate the page frome my handwritten notes.
\documentclass[preview]{standalone}

\usepackage{amsmath}
\usepackage{verbatim}
\usepackage[italian,english]{babel}
\usepackage[utf8]{inputenc}

\usepackage[basic,cat]{./Math-Symbols-List/toninus-math-symbols}
\usepackage{./Latex-Theorem/theoremtemplate}
\usepackage{./visualcat}


\begin{document} 

\begin{definition}[Natural transformation]
	\begin{tikzcd}
		\mu: F \xrightarrow[]{\cdot} G & = &
		\cat[C] \arrow[r, bend left=50, "F"{name=U, above}]
		\arrow[r, bend right=50, "G"{name=D, below}]
		& \cat[D]
		\arrow[shorten <=1pt,shorten >=1pt,Rightarrow, from=U, to=D, "\mu"]
	\end{tikzcd}
	\\
	Is a collection of morphisms 
	\begin{displaymath}
		\left\lbrace \mu_X : F(X) \rightarrow G(X) \: \vert \: X \in \Obj(\cat) \right\rbrace \subset \Arr(\cat[D])
	\end{displaymath} 
	S.t. \\
	$\forall \bulk  \textcolor{red}{\exists \bulk} \textcolor{blue}{\St \bulk}$\\
	\begin{tikzcd}
		X \arrow[d, "f "]  & & 
		F(X) \arrow[d, "F(f) "'] \arrow[r, "\eta_X ", red ] \arrow[dr,"\commute",phantom,blue] & G(X) \arrow[d, "G(f) "] \\
		Y & & F(Y)\arrow[r, "\eta_Y ", red ] & G(Y)
	\end{tikzcd}
	(naturality)
\end{definition}

\end{document}

%% A tentative, eye-catchy, one-page definition of category (I would like to replicate the page frome my handwritten notes.

\documentclass[a4paper,12pt,fleqn]{scrartcl}  %Per La Stampa
\usepackage[a4paper, margin=2cm]{geometry}

\usepackage{amsmath}

\usepackage[subpreambles=true]{standalone}
\usepackage{commath}

\usepackage[italian,english]{babel}
\usepackage[utf8]{inputenc}

\usepackage{standalone}

\usepackage{graphicx}
\usepackage{hyperref}

\usepackage[basic,cat]{./Math-Symbols-List/toninus-math-symbols}
\usepackage{./Latex-Theorem/theoremtemplate}


\usepackage[toc,page]{appendix}

\usepackage{epigraph} %funny quotes


%%__________________________________
%% To transform in a package.
%%__________________________________
%%
	\newcommand{\catdef}[3]{
		\begin{displaymath}
		#1 \leadsto
		\begin{cases}
			Obj = #2\\
			Arr = #3
		\end{cases}
		\end{displaymath}
	}
	\newcommand{\smalltikzcd}[2]{
		\begin{tikzpicture}[baseline= (a).base]
			\node[scale=#1] (a) at (0,0){
				\begin{tikzcd} #2 \end{tikzcd}
			};
		\end{tikzpicture}
	}


	\newcommand{\id}{\textrm{id}}

%Diagrammatic
\usepackage{tikz-cd}
\usepackage{xcolor}
\usepackage{fdsymbol}
	% bulk of arrows and objects
	\newcommand{\bulk}{\blacklozenge}
	% such that
	\newcommand{\St}{\textrm{s.t.}}
	% id est
	\newcommand{\ie}{\textrm{\emph{i.e.}}}
	% commuting
	\newcommand{\commute}{\circledequal}%\circlearrowleft



\begin{document} 
\begin{abstract}
	The preliminary material for my talk on the category of diffeological spaces.
\end{abstract}


\begin{appendices}
\epigraph{Category theory starts with the observation that many properties of mathematical systems can be unified and simplified by a presentation with diagrams and arrows.}{Saunders Mac Lane}
\epigraph{Poetry says more with less words.\\
Mathematics too.}{Gabriele Lolli}

%------------------------------------------------------------------------------------BASIC ---------------------------------
\section{Basic definitions}
	%% A tentative, eye-catchy, one-page definition of category (I would like to replicate the page frome my handwritten notes.
\documentclass[preview]{standalone}

\usepackage{amsmath}
\usepackage{verbatim}
\usepackage[italian,english]{babel}
\usepackage[utf8]{inputenc}

\usepackage[basic,cat]{./Math-Symbols-List/toninus-math-symbols}
\usepackage{./Latex-Theorem/theoremtemplate}
\usepackage{./visualcat}


\begin{document} 
\begin{definition}[Category ]
Category is a tuple $($data,structure,axioms$)$.
\begin{itemize}
\item \emph{data}
	\begin{displaymath}
		\cat \leadsto
		\begin{cases}
			\Obj(\cat) : 
			    \text{\parbox{0.65\textwidth}{ collection of items called \emph{objects of $\cat$}}}
			\\
			\biggr\lbrace \hom_{\cat} (A,B) \biggr\rbrace_{A,B \in \Obj(\cat)} : 
				\text{\parbox{0.45\textwidth}{parametrized family of collections of items  called \emph{morphisms} or \emph{arrows}}}
		\end{cases}
	\end{displaymath}

\item Structure:
	(Composition of arrows)
	\begin{displaymath}
		\circ : \hom_{\cat}(A,B) \times \hom_{\cat}(B,C) \rightarrow \hom_{\cat} (A,C)
	\end{displaymath}
	Diagrammatically \footnote{This is more properly a diagrammatic definition of  $\commute$}: 
	$\forall \bulk \textcolor{red}{\exists \bulk} \textcolor{blue}{\St \commute}$\\
	\begin{tikzcd}
		A \arrow[d, "f"] \arrow[dd, bend left=60,red, "gf\textcolor{blue}{ = f \circ g}"] 	& \\
		B \arrow[d, "g"] \arrow[r, phantom,"\commute",, very near start,blue] 	& \phantom.\\
		C 																& 
	\end{tikzcd}
\item Axioms:
	\begin{itemize}
		\item \emph{Identity axiom}:
			$\forall \bulk \textcolor{red}{\exists \bulk} \textcolor{blue}{\St \commute}$\\
			\begin{tikzcd}
				A \arrow[rr, "\id_A",red] \arrow[rrdd, "f", bend right]& & A \arrow[dd, "f"] \arrow[rrdd, "f", bend left] & & \\
				 & |[blue]|\commute & & |[blue]|\commute & \\
				 & &  B \arrow[rr, "\id_B",red] & & B
			\end{tikzcd}
		\item \emph{associativity axiom}
			$\forall \bulk \textcolor{green}{h (gf) = (hg) f}$			\\
			\begin{tikzcd}
				A \arrow[r,"f"] \arrow[rr,"gf",bend left=40, blue] \arrow[rrr,"h(gf)",bend left=80, green] \arrow[rrr,"(hg)f",bend right=80, green] &
				B \arrow[r,"g"] \arrow[rr,"hg",bend right=40, blue] &
				C \arrow[r,"h"] &				
				D  
			\end{tikzcd}
	\end{itemize}
	
\end{itemize}
\end{definition}
\begin{notation}
	$\Arr(\cat) = \bigcup\limits_{A,B \in \Obj(\cat)}\hom_{\cat} (A,B) $
\end{notation}

\end{document}

We can not delve here on the foundational issues related to this definitions. Let us simple look at it a triple composed of \emph{data}, \emph{structure}, \emph{axiom}.
\\
One of these problem is the problem of "size". We use the term "collections" when we want to remain vague about the axiomatic framework of sets considered.
We use "mapping" when we want to consider correspondences between collections.
\\
Words like "set" and "function" are reserved to objects and morhpisms in the set category.


\paragraph{How to compare cat?}
	%% A tentative, eye-catchy, one-page definition of category (I would like to replicate the page frome my handwritten notes.
\documentclass[preview]{standalone}

\usepackage{amsmath}
\usepackage{verbatim}
\usepackage[italian,english]{babel}
\usepackage[utf8]{inputenc}

\usepackage[basic,cat]{./Math-Symbols-List/toninus-math-symbols}
\usepackage{./Latex-Theorem/theoremtemplate}
\usepackage{./visualcat}


\begin{document} 

\begin{definition}[Functor]
	A functor $F: \cat[C] \rightarrow \cat[D] $ is a pair of correpondences 
	\begin{displaymath}
		F: \Obj(\cat[C]) \rightarrow \Obj(\cat[D])
	\end{displaymath}
	\begin{displaymath}
		F_{A,\, B} : \hom_{\cat[C]}(A,B) \rightarrow \hom_{\cat[D]}(F(A),F(B))		\qquad \forall A,B \in \Obj(\cat[C]) 
	\end{displaymath}
	
	\begin{center}
		\begin{tikzcd}
			A \arrow[d,"f"] & & F(A) \arrow[d,"F(f)",blue, shift left=1ex] \\
			B								& & F(B) \arrow[u,"F(f)",green, shift left=1ex]
		\end{tikzcd}
		(\textcolor{green}{controvariant}) (\textcolor{blue}{covariant})	
	\end{center}

	
	\St
	\begin{itemize}
		\item 
			\parbox{0.35\textwidth}{		
				\emph{identity axiom}
				$$\forall \bulk  \textcolor{blue}{\Rightarrow \commute}$$
			}
			\begin{tikzcd}
				A \arrow[d,"\id_A"] & & F(A)\arrow[d, phantom,"\commute",,blue]  \arrow[d, bend right=60, "F(\id_A)"'] \arrow[d, bend left=60, "\id_{F(A)}"] \\
				A & & F(A)
			\end{tikzcd}
		\item 
			\parbox{0.35\textwidth}{
				\emph{associativity axiom}
				$$\forall \bulk  \textcolor{blue}{\Rightarrow  \commute}$$
			}
			\begin{tikzcd}
		A \arrow[d, "f"] \arrow[dd, bend right=60, "gf "] 	& & F(A) \arrow[d, "F(f)"] \arrow[dd, bend left=60, "F(gf) "] &	 \\
		B \arrow[d, "g"] & &  F(B) \arrow[d, "F(g)"] \arrow[r, phantom,"\commute",, very near start,blue]  & \phantom.\\
		C 	& & F(C)	&
			\end{tikzcd}
	\end{itemize}
\end{definition}
\end{document}


	\subparagraph{Basic Classification:}
	%% A tentative, eye-catchy, one-page definition of category (I would like to replicate the page from my handwritten notes.)
\documentclass[preview]{standalone}

\usepackage{amsmath}
\usepackage{verbatim}
\usepackage[italian,english]{babel}
\usepackage[utf8]{inputenc}

\usepackage[basic,cat]{./Math-Symbols-List/toninus-math-symbols}
\usepackage{./Latex-Theorem/theoremtemplate}
\usepackage{./visualcat}


\begin{document} 

\begin{definition}[Faithful Functor]
	\parbox{0.35\textwidth}{
		$F_{A,\; B} $ is injective. \ie : 
		$$\forall \bulk  \textcolor{blue}{s.t. \bulk} \textcolor{red}{\Rightarrow \bulk}$$
	}
	\begin{tikzcd}
		A \arrow[d, bend right=60, "f "'] \arrow[d, bend left=60, "g "] \arrow[d, phantom,"\commute",,red] & & 
		F(A) \arrow[d, bend right=60, "F(f) "'] \arrow[d, bend left=60, "F(g) "] \arrow[d, phantom,"\commute",,blue] \\
		B & & F(B)
	\end{tikzcd}
\end{definition}
%
\begin{definition}[Full Functor]
	\parbox{0.35\textwidth}{
		$F_{A,\; B} $ is surjective. \ie :
		$$\forall \bulk  \textcolor{blue}{\exists \bulk} \textcolor{red}{\St \bulk}$$
	}
	\begin{tikzcd}
		A \arrow[d, "f ",blue]  & &  
		F(A) \arrow[d, bend right=60, "F(f) "', blue] \arrow[d, bend left=60, "\eta "] \arrow[d, phantom,"\commute",,red] \\
		B & & F(B)
	\end{tikzcd}
\end{definition}
%
\begin{definition}[Costant Functor]
	$ \bigtriangleup_u : \cat \rightarrow \cat[D] $ s.t.
	\begin{tikzcd}
		A \arrow[d, "f "]  & &  
		\bigtriangleup_u(A) = u \arrow[d, "\bigtriangleup_u(f) = \id_u"]  \\
		B & & \bigtriangleup_u(A) = u
	\end{tikzcd}
\end{definition}
%
\begin{definition}[Iso-reflecting]
	\parbox{0.35\textwidth}{
		$$\forall \bulk  \textcolor{blue}{\Rightarrow \commute}$$
	}

	\begin{tikzcd}
		A \arrow[d, "f ", bend right=60]  & & F(A) \arrow[d, "F(f) "', bend right=60] \arrow[drr, phantom,"\commute",blue] & & \\
		B \arrow[u, "f^{-1}", bend right=60] & & F(B) \arrow[u, "F(f^{-1})"', bend right=60] \arrow[u, "F(f)^{-1}"', bend right=90, blue, looseness=5] & & \phantom{A}
	\end{tikzcd}
	\\ or \\
	\begin{tikzcd}
		A \arrow[d, "f "']   \arrow[dd, "id_A"', bend right=60]& & F(A) \arrow[d, "F(f) "'] \arrow[dd, "id_{F(A)} ", bend left=60]  & \\
		B \arrow[d, "f^{-1}"'] & & F(B)  \arrow[d, "F(f^{-1}) "'] \arrow[r, phantom,"\commute",blue, near start]&\phantom{A}\\
		A & & F(A) &
	\end{tikzcd}
	
\end{definition}

\begin{proposition}
	Given F a functor
	$$ \textrm{Fully Faithfull} \Leftrightarrow \textrm{Iso-reflecting}$$
\end{proposition}

\end{document}


\subparagraph{How to compare Functors?}
	%% A tentative, eye-catchy, one-page definition of category (I would like to replicate the page frome my handwritten notes.
\documentclass[preview]{standalone}

\usepackage{amsmath}
\usepackage{verbatim}
\usepackage[italian,english]{babel}
\usepackage[utf8]{inputenc}

\usepackage[basic,cat]{./Math-Symbols-List/toninus-math-symbols}
\usepackage{./Latex-Theorem/theoremtemplate}
\usepackage{./visualcat}


\begin{document} 

\begin{definition}[Natural transformation]
	\begin{tikzcd}
		\mu: F \xrightarrow[]{\cdot} G & = &
		\cat[C] \arrow[r, bend left=50, "F"{name=U, above}]
		\arrow[r, bend right=50, "G"{name=D, below}]
		& \cat[D]
		\arrow[shorten <=1pt,shorten >=1pt,Rightarrow, from=U, to=D, "\mu"]
	\end{tikzcd}
	\\
	Is a collection of morphisms 
	\begin{displaymath}
		\left\lbrace \mu_X : F(X) \rightarrow G(X) \: \vert \: X \in \Obj(\cat) \right\rbrace \subset \Arr(\cat[D])
	\end{displaymath} 
	S.t. \\
	$\forall \bulk  \textcolor{red}{\exists \bulk} \textcolor{blue}{\St \bulk}$\\
	\begin{tikzcd}
		X \arrow[d, "f "]  & & 
		F(X) \arrow[d, "F(f) "'] \arrow[r, "\eta_X ", red ] \arrow[dr,"\commute",phantom,blue] & G(X) \arrow[d, "G(f) "] \\
		Y & & F(Y)\arrow[r, "\eta_Y ", red ] & G(Y)
	\end{tikzcd}
	(naturality)
\end{definition}

\end{document}


%------------------------------------------------------------------------------------Limits ---------------------------------
\newpage
\section{Limits}
	%% A tentative, eye-catchy, one-page definition of category (I would like to replicate the page frome my handwritten notes.
\documentclass[preview]{standalone}

\usepackage{amsmath}
\usepackage{verbatim}
\usepackage[italian,english]{babel}
\usepackage[utf8]{inputenc}

\usepackage[basic,cat]{./Math-Symbols-List/toninus-math-symbols}
\usepackage{./Latex-Theorem/theoremtemplate}
\usepackage{./visualcat}

\def\trilim{
	\tikz[baseline=.1ex]{
		\fill (0,1.75ex) coordinate (A);
		\fill (2.5ex,1.75ex) coordinate (B);
		\fill (1.25ex,-0.5ex) coordinate (C);	
		\draw[green] (1.25ex,1ex) node {$\commute$};
		\draw[black] (A) -- (B);
		\draw[green] (B) -- (C);
		\draw[red] (C) -- (A);
	}
}

\def\tricolim{
	\tikz[baseline=.1ex]{
		\fill (0,-0.5ex) coordinate (A);
		\fill (2.5ex,-0.5ex) coordinate (B);
		\fill (1.25ex,1.75ex) coordinate (C);
		\draw[green] (1.25ex,0.25ex) node {$\commute$};
		\draw[black] (A) -- (B);
		\draw[green] (B) -- (C);
		\draw[red] (C) -- (A);
	}
}


\def\TriLimit{
	\tikz[baseline=.1ex]{
		\fill (0,-0.5ex) coordinate (A);
		\fill (0,2ex) coordinate (B);
		\fill (-2.25ex,0.75ex) coordinate (C);
		\draw[orange] (-0.75ex,0.75ex) node {$\commute$};
		\draw[orange] (A) -- (B);
		\draw[red] (B) -- (C) ;
		\draw[blue] (C) -- (A) ;
	}
}



\begin{document}
% For every picture that defines or uses external nodes, you'll have to
% apply the 'remember picture' style. To avoid some typing, we'll apply
% the style to all pictures.
\tikzstyle{every picture}+=[remember picture]

% By default all math in TikZ nodes are set in inline mode. Change this to
% displaystyle so that we don't get small fractions.
\everymath{\displaystyle}



Let's begin from the dry abstract nonsense:
\begin{definition}[Diagram on $\cat$]
	Functor $D: \cat[I] \rightarrow \cat$ from a small category $\cat[I]$ to $\cat[C]$.
\end{definition}
Basically it is  a graph composed of objects and arrows in $\cat$.

% first column
\begin{minipage}[t]{0.5\textwidth}
	\begin{definition}[Cone over diagram $D$]
		\begin{displaymath}
		 (
		 \tikz[baseline]{
		 			\node[fill=blue!20,anchor=base] (t1)
		            {$ V$};
		        } 
			,
			 \tikz[baseline]{
		            \node[fill=blue!20,anchor=base] (t2)
		            {$ \{\nu_i:V \rightarrow D_i\}_{i\in \cat[I]} $};
			}
		)
		\end{displaymath}
		\tikzstyle{na} = [baseline=-.5ex]
		\begin{itemize}
		    \item \tikz[na] \node[coordinate] (n1) {};
		    	Object $\in \cat$
		    \item Family of morphisms of $\cat$
		        \tikz[na]\node [coordinate] (n2) {};
		\end{itemize}
		% Now it's time to draw some edges between the global nodes. Note that we have to apply the 'overlay' style.
		\begin{tikzpicture}[overlay]
		        \path[->] (n1) edge [bend left] (t1);
		        \path[->] (n2) edge [bend right] (t2);
		        %\path[->] (n3) edge [out=0, in=-90] (t3);
		\end{tikzpicture}
		\St
		
		$\forall \left( i \xrightarrow[]{f} j \right) \in \cat[I]$
		\begin{tikzcd}
			& \textcolor{red}{V} \ar[dl,red,"\nu_i"'] \ar[d,phantom,blue,"\commute"] \ar[dr,red,"\nu_j"]& \\
			D_i \ar[rr,"D(f)"']& \phantom. & D_j 
		\end{tikzcd}
		I.e. it is a n.t. $ \nu : \Delta_V \xrightarrow{\cdot} \cat[D]$
		\begin{tikzcd}
			i \ar[d,"f"] & & \textcolor{red}{V} \ar[d,equal,red] \ar[r,red,"\nu_i"] \ar[dr,phantom,blue,"\commute"]& D_i \ar[d,"D(f)"]\\
			j			   & & \textcolor{red}{V} \ar[r,red,"\nu_j"'] & D_j
		\end{tikzcd}
	\end{definition}
\end{minipage}
\vrule{}
\begin{minipage}[t]{0.5\textwidth}
	\begin{definition}[CoCone over diagram $D$]
		\begin{displaymath}
		 (
		 \tikz[baseline]{
		 			\node[fill=blue!20,anchor=base] (t1)
		            {$ W$};
		        } 
			,
			 \tikz[baseline]{
		            \node[fill=blue!20,anchor=base] (t2)
		            {$ \{\omega_i:D_i \rightarrow W\}_{i\in \cat[I]} $};
			}
		)
		\end{displaymath}
		\tikzstyle{na} = [baseline=-.5ex]
		\begin{itemize}
		    \item \tikz[na] \node[coordinate] (n1) {};
		    	Object $\in \cat$
		    \item Family of morphisms of $\cat$
		        \tikz[na]\node [coordinate] (n2) {};
		\end{itemize}
		\begin{tikzpicture}[overlay]
		        \path[->] (n1) edge [bend left] (t1);
		        \path[->] (n2) edge [bend right] (t2);
		        %\path[->] (n3) edge [out=0, in=-90] (t3);
		\end{tikzpicture}
		\St
		
		$\forall \left( i \xrightarrow[]{f} j \right) \in \cat[I]$
		\begin{tikzcd}
			D_i \ar[rr,"D(f)"] \ar[dr,red,"\omega_i"']& \phantom. & D_j \ar[dl,red,"\omega_j"]\\
			& \textcolor{red}{W}  \ar[u,phantom,blue,"\commute"] & 
		\end{tikzcd}
		I.e. it is a n.t. $ \omega : \cat[D] \xrightarrow{\cdot} \Delta_W$
		\begin{tikzcd}
			i \ar[d,"f"] & & D_i \ar[d,"D(f)"'] \ar[r,red,"\omega_i"] \ar[dr,phantom,blue,"\commute"]& \textcolor{red}{W}\ar[d,equal,red] \\
			j			   & & D_j \ar[r,red,"\omega_j"'] & \textcolor{red}{W}
		\end{tikzcd}
	\end{definition}
\end{minipage}
	\begin{center}
		\begin{tikzcd}[row sep= 5pt]%small]
			e.g.:& &   \textcolor{green}{V} \ar[ddl, green, bend right]\ar[dd, green] \ar[dr, green, bend left] \ar[dddr, green,bend left]  &   & 
				\textcolor{green}{\textrm{\small cone}}    \\
			& &		   & D_3  & \\
			& D_1 \ar[r] & D_2 \ar[ru] \ar[rd]& \phantom. & \textrm{\small diagram}\\ 
			& &		   & D_4 \\
			& &   \textcolor{red}{W} \ar[uul, red, bend left]\ar[uu, red] \ar[ur, red, bend right] \ar[uuur, red, bend right]  &   &
				\textcolor{red}{\textrm{\small co-cone}}    
		\end{tikzcd}
	\end{center}

% first column
\begin{minipage}[t]{0.5\textwidth}
	\begin{definition}[Limit of diagram $D$]
		It is an Universal cone over $D$.
		$$\textcolor{red}{cone \bulk} s.t. \textcolor{blue}{\forall cone \bulk} \quad \textcolor{orange}{\exists! \bulk} \St \TriLimit$$	
		\begin{tikzcd}[row sep= small]
		i \ar[dd,"f"] & & D_i  \ar[dd,"D(f)"']& \\
		 & & \phantom. \ar[r,phantom,red,"\commute"] \ar[rdd,phantom,blue,"\commute"]& \textcolor{red}{V} \ar[lu,red] \ar[ld,red] \\
		j & & D_j & \phantom. \ar[l,phantom,near start, orange,"\commute"]\\
		& & & \textcolor{blue}{V'} \ar[uuul,blue] \ar[ul,blue] \ar[uu,orange,"\exists!"']
		\end{tikzcd}		
		
	\end{definition}
\end{minipage}
\vrule{}
\begin{minipage}[t]{0.5\textwidth}
	\begin{definition}[Co-Limit of diagram $D$]
		It is an Universal co-cone over $D$.
		$$\textcolor{red}{cocone \bulk} s.t. \textcolor{blue}{\forall cone \bulk} \quad \textcolor{orange}{\exists! \bulk} \St \TriLimit$$	
		\begin{tikzcd}[row sep= small]
		i \ar[dd,"f"] & & D_i \ar[rd,red] \ar[rddd,blue] \ar[dd,"D(f)"']& \\
		 & & \phantom. \ar[r,phantom,red,"\commute"] \ar[rdd,phantom,blue,"\commute"] & \textcolor{red}{W} \ar[dd,orange,"\exists!"] \\
		j & & D_j \ar[ru,red]  \ar[rd,blue]& \phantom. \ar[l,phantom,near start, orange,"\commute"]\\
		& & & \textcolor{blue}{W'}
		\end{tikzcd}	

	\end{definition}
\end{minipage}
\end{document}
	The best way to grasp the meaning (...) to understand what is a limit it is through examples. The model category is $\cat[Set]$ which turn out to be complete and co-complete.
	%% A tentative, eye-catchy, one-page definition of category (I would like to replicate the page frome my handwritten notes.
\documentclass[preview]{standalone}

\usepackage{amsmath}
\usepackage{verbatim}
\usepackage[italian,english]{babel}
\usepackage[utf8]{inputenc}

\usepackage[basic,cat]{./Math-Symbols-List/toninus-math-symbols}
\usepackage{./Latex-Theorem/theoremtemplate}
\usepackage{./visualcat}
\usepackage{pdflscape}



\begin{document}
%
The best way to grasp the meaning (...) to understand what is a limit it is through examples. The model category is $\cat[Set]$ which turn out to be complete and co-complete.

\begin{definition}[Terminal Object]
	Is a limit on the empty diagram.\\
	$\textcolor{red}{T\in \Obj(\cat)} \St 
	 \textcolor{blue}{\forall X \in \Obj	(\cat)}
	 \textcolor{green}{\exists! X\rightarrow T}$\\
	\begin{tikzcd}
		\textcolor{blue}{X} \ar[d,green] \\
		\textcolor{red}{T} 
	\end{tikzcd}
\end{definition}
%
\begin{definition}[Initial Object]
	\begin{tikzcd}
		\textcolor{red}{I} \ar[d,green] \\
		\textcolor{blue}{X} 
	\end{tikzcd}
\end{definition}
%
\begin{definition}[Product]
	\begin{tikzcd}
		&\textcolor{blue}{C} \ar[ddl, bend right, blue] \ar[ddr, bend left, blue] \ar[d,green] 
		\ar[ddl,phantom,green, "\commute"] \ar[ddr,phantom,green, "\commute"]& \\
		&\textcolor{red}{A \times B} \ar[dl,red] \ar[dr,red]& \\
		A & & B \\
	\end{tikzcd}
\end{definition}
%
\begin{definition}[Co-Product]
	\begin{tikzcd}
		\textcolor{red} A \ar[dr,red] \ar[ddr, bend right, blue] & & \textcolor{red} B\ar[dl,red] \ar[ddl, bend left, blue]\\
		&\textcolor{red}{A \times B}  \ar[d,green] & \\
		&\textcolor{blue}{C} \ar[uul,phantom,green, "\commute"] \ar[uur,phantom,green, "\commute"]& 
	\end{tikzcd}
\end{definition}
%
\begin{definition}[Equalizer]
	\begin{tikzcd}
		 & & B \ar[dd,equal] \\
		\textcolor{red} E \ar[r,red,"e"] \ar[rru,red, bend left] \ar[rrd,red, bend right] \ar[rrd,phantom,red, "\commute"] \ar[rru,phantom,red, "\commute"]& 
		A \ar[ur,"f"] \ar[dr,"g"'] \ar[r,phantom, "\commute"]& \phantom. \\
		 & & B 
	\end{tikzcd}
\end{definition}
%
\begin{definition}[Co-Equalizer]
	\begin{tikzcd}
		A   \ar[dd,equal] \ar[rd,"f"] \ar[rrd,red, bend left] \ar[rrd,phantom,red, "\commute"]& & \\
		\phantom. \ar[r,phantom, "\commute"] & B \ar[r,red,"q"] & \textcolor{red} Q \\
		B \ar[ru,"g"'] \ar[rru,red, bend right] \ar[rru,phantom,red, "\commute"]& & \\
	\end{tikzcd}
\end{definition}
%
\begin{definition}[Pull-Back]
	\begin{tikzcd}
		\textcolor{green} U \ar[rd,orange,"!"]  \ar[rdd,green,bend right] \ar[rrd,green,bend left]& & \\
		& \textcolor{red} S \ar[rd,red,phantom,"\commute"] \ar[r,red] \ar[d,red]  \arrow[dr, phantom, "\ulcorner", very near start]
& A \ar[d] \\
		& B \ar[r] & C
	\end{tikzcd}
\end{definition}
%
\begin{definition}[Push-Out]
	\begin{tikzcd}
		C \ar[r] \ar[d] \ar[rd,red,phantom,"\commute"]& A \ar[d,red] \ar[rdd,green,bend left]& \\
		B  \ar[r,red] \ar[rrd,green,bend right] & \textcolor{red}O \arrow[ul, phantom, "\ulcorner", very near start] \ar[rd,orange,"!"] & \\
		& & \textcolor{green} U		
	%	 U   \ar[rdd,green,bend right] \ar[rrd,green,bend left]& & \\
	%	& } S  \arrow[dr, phantom, "\ulcorner", very near start] & A \ar[d] \\
	%	& B \ar[r] & C
	\end{tikzcd}
\end{definition}
%
So, a limit is a concept pertaing to a specific configuration of objects and arrows (the image of the diagram $D$ in $\cat$).
\\
We reserve a special name for categories such that limit exists for all diagram of a certain shape regardless of the specificity of vertex and arrows (i.e. depending only on the shape of the diagrams).
%
\begin{notation}
	\begin{displaymath}
		D: \cat[I] \rightarrow	\cat
	\end{displaymath}
	A diagram is \emph{finite} if $\Obj(\cat[I])$ is a discrete set.
	\\
	Limits on a finite diagram are called \emph{finite limits}.
\end{notation}
%
\begin{definition}[(finitely) (Co-)Complete Category]
	$\cat$ \St all (finite) (Co-)Limit exists.
\end{definition}
%
\begin{proposition}
T.f.s.a.e.:
\begin{itemize}
	\item $\cat$ has all (finite) limits / (finite) colimits
	\item $\cat$ has all (finite) products and equalizers / (finite) co-products and co-equalizers
	\item $cat$ has all (binary) pullbacks and terminal object / (binary) pushout and initial object
\end{itemize}

\end{proposition}

\end{document}

\newpage
\section{Classification of Morphisms}
	In any $\cat$ there is a basic classification of morphisms $f: C \rightarrow D$.\\
	

	%% A tentative, eye-catchy, one-page definition of category (I would like to replicate the page frome my handwritten notes.
\documentclass[preview]{standalone}

\usepackage{amsmath}
\usepackage{verbatim}
\usepackage[italian,english]{babel}
\usepackage[utf8]{inputenc}

\usepackage[basic,cat]{./Math-Symbols-List/toninus-math-symbols}
\usepackage{./Latex-Theorem/theoremtemplate}
\usepackage{./visualcat}

\usepackage{tikz}
\def\tria{
	\tikz[baseline=.1ex]{
		\fill (0,-0.5ex) coordinate (A);
		\fill (0,2ex) coordinate (B);
		\fill (2.25ex,0.75ex) coordinate (C);
		\draw[green] (0.75ex,0.75ex) node {$\commute$};
		\draw[blue] (A) -- (B);
		\draw[green] (A) -- (C) -- (B);
	}
}

\def\trib{
	\tikz[baseline=.1ex]{
		\fill (0,-0.5ex) coordinate (A);
		\fill (0,2ex) coordinate (B);
		\fill (2.25ex,0.75ex) coordinate (C);
		\draw[blue] (0.75ex,0.75ex) node {$\commute$};
		\draw[blue] (A) -- (B);
		\draw[blue] (A) -- (C) -- (B);
	}
}

\def\cotria{
	\tikz[baseline=.1ex]{
		\fill (0,-0.5ex) coordinate (A);
		\fill (0,2ex) coordinate (B);
		\fill (-2.25ex,0.75ex) coordinate (C);
		\draw[green] (-0.75ex,0.75ex) node {$\commute$};
		\draw[blue] (A) -- (B);
		\draw[green] (A) -- (C) -- (B);
	}
}

\def\cotrib{
	\tikz[baseline=.1ex]{
		\fill (0,-0.5ex) coordinate (A);
		\fill (0,2ex) coordinate (B);
		\fill (-2.25ex,0.75ex) coordinate (C);
		\draw[blue] (-0.75ex,0.75ex) node {$\commute$};
		\draw[blue] (A) -- (B);
		\draw[blue] (A) -- (C) -- (B);
	}
}

\begin{document} 
% first column
\begin{minipage}[t]{0.5\textwidth}
	\begin{definition}[Monomorphism (mono or monic arrow)]
		\emph{"left cancellative arrow"}
		$\textcolor{red}{\bulk} s.t. \textcolor{blue}{\forall \bulk} \quad \tria \Rightarrow \trib$\\
		%
		\begin{tikzcd}
			 \textcolor{blue}{B} \ar[dd,blue,equal] \ar[dr,blue,"g_1"] \arrow[rrd, bend left=30, "fg_1",green] & & \\
			\phantom. & \textcolor{red}{C}  \ar[r,red,"f",tail]& \textcolor{red}{D}  \\
			\textcolor{blue}{B}  \ar[ur,blue,"g_2"] \arrow[rru, bend right=30, "fg_2",green]& & 
		\end{tikzcd}
		\footnote{$(f g_1 = f g_2 ) \Rightarrow ( g_1 = g_2 )$}
	\end{definition}
	(Hom functor point of view:)
	\begin{displaymath}
		f \textrm{mono} \Leftrightarrow H^B(f) \textrm{injective} \quad \forall B \in \cat
	\end{displaymath}
	%
	\begin{tikzpicture}[baseline= (a).base]
	\node[scale=.75] (a) at (0,0){
		\begin{tikzcd}
			C \ar[d,"f"] & & \hom(B,C) \ar[d, "H^B(f) = f(-)"]\ar[r,phantom,"\in"] & g_1 \ar[d,mapsto]\ar[r,phantom,','] & g_2 \ar[d,mapsto] \\
			D				 & & \hom(B,D) \ar[r,phantom,"\in"] & f g_1 \ar[r,phantom,"="] & f g_2 	
		\end{tikzcd}
	};
	\end{tikzpicture}
	%
	\begin{example}
	(Mono "generalize" (there are nuance of monomorphism) the concept of injective map)\\
	Monomorphisms in $Set$ are injective functions. (It's straightforward from the definition via hom functors taking $B= {\ast}$)
	\begin{tikzpicture}[baseline= (a).base]
	\node[scale=.75] (a) at (0,0){
		\begin{tikzcd}
			\{\ast\} \ar[dd,equal] \ar[dr,"c_1"] & & \\
			& C \ar[r,tail,"f"] & D \\
			\{\ast\} \ar[ur,"c_2"]& &
		\end{tikzcd}
	};
	\end{tikzpicture}
	\\
	Mono means:
	\begin{displaymath}
		f(c1) = f(c2) \Rightarrow c1=c2
	\end{displaymath}
	that is injectivity
	\end{example}
 
\end{minipage}
\vrule{}
\begin{minipage}[t]{0.5\textwidth}
	\begin{definition}[Epimorphism (epi or epic arrow)]
		\emph{"right cancellative arrow"}
		$\textcolor{red}{\bulk} s.t. \textcolor{blue}{\forall \bulk} \quad \cotria \Rightarrow \cotrib$
		
		\begin{tikzcd}
			& & \textcolor{blue}{E} \ar[dd,blue,equal] \\
			\textcolor{red}{C} \ar[r,red,"f",twoheadrightarrow] \arrow[rru, bend left=30, "h_1 f",green] \arrow[rrd, bend right=30, "h_2 f",green]& \textcolor{red}{D} \ar[ur,blue,"h_1"] \ar[dr,blue,"h_2"] & \\
			& & \textcolor{blue}{E}
		\end{tikzcd}
		\footnote{$(h_1 f = h_2 f ) \Rightarrow ( h_1 = h_2 )$}
	\end{definition}

	(Hom functor point of view:)
	\begin{displaymath}
		f \textrm{epi} \Leftrightarrow H_E(f) \textrm{injective} \quad \forall E \in \cat
	\end{displaymath}	
			\begin{tikzpicture}[baseline= (a).base]
			\node[scale=.75] (a) at (0,0){
				\begin{tikzcd}
					C \ar[d,"f"] & & \hom(C,E) \ar[r,phantom,"\in"] & h_1 f \ar[r,phantom,"="] & h_2 f  \\
					D				 & & \hom(D,E) \ar[u, "H_E(f) = (-)f"]  \ar[r,phantom,"\in"] & h_1 \ar[u,mapsto] \ar[r,phantom,','] & h_2 \ar[u,mapsto]	
				\end{tikzcd}
			};
			\end{tikzpicture}
	%
	\begin{example}
	(Epi "generalizes" (there are nuance of monomorphism) the concept of surjective map)\\
	Epimorphisms in $Set$ are surjective functions. \\
	Consider
	\begin{tikzpicture}[baseline= (a).base]
	\node[scale=.75] (a) at (0,0){
		\begin{tikzcd}
			& & \{0,1\} \ar[dd,equal]\\
		X	\ar[r,"f",twoheadrightarrow]&Y \ar[ur,"\chi_{f(x)}"]\ar[dr,"1"] & \\
			&	 & \{0,1\}
		\end{tikzcd}
	};
	\end{tikzpicture}
	\\
	where $\chi_{f(x)}$ is the characteristic function on the range $f(X) \subset Y$ and $1$ is the constant function.\\
	Epi means:
	\begin{displaymath}
		\chi_{f(x)} = 1 \Rightarrow \chi_{f(X)} = 1 \Rightarrow Y = f(X)
	\end{displaymath}
	that is surjectivity
	\end{example}
	
\end{minipage}






\end{document}


\newpage
\section{Subobjects}

\newpage
\section{Cartesian Closedness}



\section{Eternal TODO}
	\documentclass[preview]{standalone}

\usepackage{amsmath}
\usepackage{verbatim}

\usepackage[basic,cat]{./Math-Symbols-List/toninus-math-symbols}
\usepackage{./Latex-Theorem/theoremtemplate}
\usepackage{./visualcat}

\begin{document}
	%Category
	\begin{displaymath}
		\frac{\cat}{B}
	\end{displaymath}
	% Objects
	\begin{displaymath}
		Obj = \{ \pi: E \rightarrow M \} = \{ \pi \in \Arr(\cat) \vert \cod(\pi)=B\}
	\end{displaymath}
	% Arrows
	\begin{displaymath}
	 \hom_{\dfrac{\cat}{B}}(\pi_1,\pi_2) = \left\{\phi \in \hom_{\cat}(\cod(\pi_1),\cod(\pi_2)) \vert 
		\begin{tikzpicture}[baseline= (a).base]
			\node[scale=.75] (a) at (0,0){
				\begin{tikzcd} E \ar[rr,"\phi"] \ar[dr,"\pi_1"] &  \phantom.\arrow[d, phantom,"\commute"] & E' \ar[dl,"\pi_2"] \\ & M & \end{tikzcd}
			};
		\end{tikzpicture}
	\right\}
	\end{displaymath}

\end{document}
	
	



\end{appendices}
			\nocite{*}
			\bibliographystyle{plain}
			\bibliography{biblio}

\end{document}

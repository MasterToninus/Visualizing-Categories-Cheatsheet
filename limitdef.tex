%% A tentative, eye-catchy, one-page definition of category (I would like to replicate the page frome my handwritten notes.
\documentclass[preview]{standalone}

\usepackage{amsmath}
\usepackage{verbatim}
\usepackage[italian,english]{babel}
\usepackage[utf8]{inputenc}

\usepackage[basic,cat]{./Math-Symbols-List/toninus-math-symbols}
\usepackage{./Latex-Theorem/theoremtemplate}
\usepackage{./visualcat}

\def\trilim{
	\tikz[baseline=.1ex]{
		\fill (0,1.75ex) coordinate (A);
		\fill (2.5ex,1.75ex) coordinate (B);
		\fill (1.25ex,-0.5ex) coordinate (C);	
		\draw[green] (1.25ex,1ex) node {$\commute$};
		\draw[black] (A) -- (B);
		\draw[green] (B) -- (C);
		\draw[red] (C) -- (A);
	}
}

\def\tricolim{
	\tikz[baseline=.1ex]{
		\fill (0,-0.5ex) coordinate (A);
		\fill (2.5ex,-0.5ex) coordinate (B);
		\fill (1.25ex,1.75ex) coordinate (C);
		\draw[green] (1.25ex,0.25ex) node {$\commute$};
		\draw[black] (A) -- (B);
		\draw[green] (B) -- (C);
		\draw[red] (C) -- (A);
	}
}


\def\TriLimit{
	\tikz[baseline=.1ex]{
		\fill (0,-0.5ex) coordinate (A);
		\fill (0,2ex) coordinate (B);
		\fill (-2.25ex,0.75ex) coordinate (C);
		\draw[orange] (-0.75ex,0.75ex) node {$\commute$};
		\draw[orange] (A) -- (B);
		\draw[red] (B) -- (C) ;
		\draw[blue] (C) -- (A) ;
	}
}



\begin{document}
% For every picture that defines or uses external nodes, you'll have to
% apply the 'remember picture' style. To avoid some typing, we'll apply
% the style to all pictures.
\tikzstyle{every picture}+=[remember picture]

% By default all math in TikZ nodes are set in inline mode. Change this to
% displaystyle so that we don't get small fractions.
\everymath{\displaystyle}



Let's begin from the dry abstract nonsense:
\begin{definition}[Diagram on $\cat$]
	Functor $D: \cat[I] \rightarrow \cat$ from a small category $\cat[I]$ to $\cat[C]$.
\end{definition}
Basically it is  a graph composed of objects and arrows in $\cat$. We call \emph{shape of $D$} the corresponding diagram.

% first column
\begin{minipage}[t]{0.5\textwidth}
	\begin{definition}[Cone over diagram $D$]
	
		\begin{displaymath}
		 (
		 \tikz[baseline]{
		 			\node[fill=blue!20,anchor=base] (t1)
		            {$ V$};
		        } 
			,
			 \tikz[baseline]{
		            \node[fill=blue!20,anchor=base] (t2)
		            {$ \{\nu_i:V \rightarrow D_i\}_{i\in \cat[I]} $};
			}
		)
		\end{displaymath}
		\tikzstyle{na} = [baseline=-.5ex]
		\begin{itemize}
		    \item[] \tikz[na] \node[coordinate] (n1) {};
		    	Object $\in \cat$
		    \item[] Family of morphisms of $\cat$
		        \tikz[na]\node [coordinate] (n2) {};
		\end{itemize}
		% Now it's time to draw some edges between the global nodes. Note that we have to apply the 'overlay' style.
		\begin{tikzpicture}[overlay]
		        \path[->] (n1) edge [bend left] (t1);
		        \path[->] (n2) edge [bend right] (t2);
		        %\path[->] (n3) edge [out=0, in=-90] (t3);
		\end{tikzpicture}
		\St
		
		\parbox{0.4\textwidth}{
			$$\forall \left( i \xrightarrow[]{f} j \right) \in \cat[I]$$
		}
		\begin{tikzcd}[sep=small]
			& \textcolor{red}{V} \ar[dl,red,"\nu_i"'] \ar[d,phantom,blue,"\commute"] \ar[dr,red,"\nu_j"]& \\
			D_i \ar[rr,"D(f)"']& \phantom. & D_j 
		\end{tikzcd} 
	\end{definition}
	\begin{lemma}
		Any cone is a n.t. $ \nu : \Delta_V \xrightarrow{\cdot} \cat[D]$
		\begin{tikzcd}
			i \ar[d,"f"] & & \textcolor{red}{V} \ar[d,equal,red] \ar[r,red,"\nu_i"] \ar[dr,phantom,blue,"\commute"]& D_i \ar[d,"D(f)"]\\
			j			   & & \textcolor{red}{V} \ar[r,red,"\nu_j"'] & D_j
		\end{tikzcd}	
	\end{lemma}
	
\end{minipage}
\vrule{}
\begin{minipage}[t]{0.5\textwidth}
	\begin{definition}[CoCone over diagram $D$]
	
		\begin{displaymath}
		 (
		 \tikz[baseline]{
		 			\node[fill=blue!20,anchor=base] (t1)
		            {$ W$};
		        } 
			,
			 \tikz[baseline]{
		            \node[fill=blue!20,anchor=base] (t2)
		            {$ \{\omega_i:D_i \rightarrow W\}_{i\in \cat[I]} $};
			}
		)
		\end{displaymath}
		\tikzstyle{na} = [baseline=-.5ex]
		\begin{itemize}
		    \item[] \tikz[na] \node[coordinate] (n1) {};
		    	Object $\in \cat$
		    \item[] Family of morphisms of $\cat$
		        \tikz[na]\node [coordinate] (n2) {};
		\end{itemize}
		\begin{tikzpicture}[overlay]
		        \path[->] (n1) edge [bend left] (t1);
		        \path[->] (n2) edge [bend right] (t2);
		        %\path[->] (n3) edge [out=0, in=-90] (t3);
		\end{tikzpicture}
		\St
		
		\parbox{0.4\textwidth}{
		 $$\forall \left( i \xrightarrow[]{f} j \right) \in \cat[I]$$
		}
		\begin{tikzcd}[sep=small]
			D_i \ar[rr,"D(f)"] \ar[dr,red,"\omega_i"']& \phantom. & D_j \ar[dl,red,"\omega_j"]\\
			& \textcolor{red}{W}  \ar[u,phantom,blue,"\commute"] & 
		\end{tikzcd}

	\end{definition}
	\begin{lemma}
		Any co-cone is a n.t. $ \omega : \cat[D] \xrightarrow{\cdot} \Delta_W$
		\begin{tikzcd}
			i \ar[d,"f"] & & D_i \ar[d,"D(f)"'] \ar[r,red,"\omega_i"] \ar[dr,phantom,blue,"\commute"]& \textcolor{red}{W}\ar[d,equal,red] \\
			j			   & & D_j \ar[r,red,"\omega_j"'] & \textcolor{red}{W}
		\end{tikzcd}
	
	\end{lemma}

\end{minipage}
	\begin{center}
		\begin{tikzcd}[row sep= 5pt]%small]
			e.g.:& &   \textcolor{green}{V} \ar[ddl, green, bend right]\ar[dd, green] \ar[dr, green, bend left] \ar[dddr, green,bend left]  &   & 
				\textcolor{green}{\textrm{\small cone}}    \\
			& &		   & D_3  & \\
			& D_1 \ar[r] & D_2 \ar[ru] \ar[rd]& \phantom. & \textrm{\small diagram}\\ 
			& &		   & D_4 \\
			& &   \textcolor{red}{W} \ar[uul, red, bend left,leftarrow]\ar[uu, red,leftarrow] \ar[ur, red, bend right,leftarrow] \ar[uuur, red, bend right,leftarrow]  &   &
				\textcolor{red}{\textrm{\small co-cone}}    
		\end{tikzcd}
	\end{center}

% first column
\begin{minipage}[t]{0.5\textwidth}
	\begin{definition}[Limit of diagram $D$]
		\emph{Universal cone over $D$}
		$$\textcolor{red}{cone \bulk} s.t. \textcolor{blue}{\forall cone \bulk} \quad \textcolor{orange}{\exists! \bulk} \St \TriLimit$$	
		\begin{tikzcd}[row sep= small]
		i \ar[dd,"f"] & & D_i  \ar[dd,"D(f)"']& \\
		 & & \phantom. \ar[r,phantom,red,"\commute"] \ar[rdd,phantom,blue,"\commute"]& \textcolor{red}{V} \ar[lu,red] \ar[ld,red] \\
		j & & D_j & \phantom. \ar[l,phantom,near start, orange,"\commute"]\\
		& & & \textcolor{blue}{V'} \ar[uuul,blue] \ar[ul,blue] \ar[uu,orange,"\exists!"']
		\end{tikzcd}		
		
	\end{definition}
\end{minipage}
\vrule{}
\begin{minipage}[t]{0.5\textwidth}
	\begin{definition}[Co-Limit of diagram $D$]
		\emph{Universal co-cone over $D$}
		$$\textcolor{red}{cocone \bulk} s.t. \textcolor{blue}{\forall cone \bulk} \quad \textcolor{orange}{\exists! \bulk} \St \TriLimit$$	
		\begin{tikzcd}[row sep= small]
		i \ar[dd,"f"] & & D_i \ar[rd,red] \ar[rddd,blue] \ar[dd,"D(f)"']& \\
		 & & \phantom. \ar[r,phantom,red,"\commute"] \ar[rdd,phantom,blue,"\commute"] & \textcolor{red}{W} \ar[dd,orange,"\exists!"] \\
		j & & D_j \ar[ru,red]  \ar[rd,blue]& \phantom. \ar[l,phantom,near start, orange,"\commute"]\\
		& & & \textcolor{blue}{W'}
		\end{tikzcd}	

	\end{definition}
\end{minipage}
\end{document}
\documentclass[preview]{standalone}
\usepackage{tikz}
\usepackage{amsmath}
\usepackage{verbatim}
\usetikzlibrary{arrows,shapes}

\usepackage{./visualcat}

\begin{document}
% For every picture that defines or uses external nodes, you'll have to
% apply the 'remember picture' style. To avoid some typing, we'll apply
% the style to all pictures.
\tikzstyle{every picture}+=[remember picture]

% By default all math in TikZ nodes are set in inline mode. Change this to
% displaystyle so that we don't get small fractions.
\everymath{\displaystyle}

\begin{displaymath}
 (,
 \tikz[baseline]{
            \node[fill=blue!20,anchor=base] (t1)
            {$ V1$};
        } 
	,
	 \tikz[baseline]{
            \node[fill=blue!20,anchor=base] (t2)
            {$ V2$};
	}
	,
	 \tikz[baseline]{
            \node[fill=blue!20,anchor=base] (t3)
            {$ V3$};
	}	 
)
\end{displaymath}
\tikzstyle{na} = [baseline=-.5ex]
\begin{itemize}
    \item Termine 1
        \tikz[na] \node[coordinate] (n1) {};
    \item Termine 1
        \tikz[na]\node [coordinate] (n2) {};
    \item Termine 1
        \tikz[na]\node [coordinate] (n3) {};
\end{itemize}

% Now it's time to draw some edges between the global nodes. Note that we
% have to apply the 'overlay' style.
\begin{tikzpicture}[overlay]
        \path[->] (n1) edge [bend left] (t1);
        \path[->] (n2) edge [bend right] (t2);
        \path[->] (n3) edge [out=0, in=-90] (t3);
\end{tikzpicture}

\end{document}